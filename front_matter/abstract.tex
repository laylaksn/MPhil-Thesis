\thispagestyle{plain}
    \LARGE
    \textbf{Abstract}
    
    \vspace{2cm}
    \normalsize
    Calcium ($Ca^{2+}$) signalling plays a crucial, diverse role in the body \shortcite{Berridge}. We focus on the role of $Ca^{2+}$ signalling in the fertilisation of mammalian eggs \shortcite{Sanders2018}.
    Many models of $Ca^{2+}$ signalling {rely on inaccurate dynamics of the inositol 1,4,5-triphosphate receptor ($IP_3R$)} on the Endoplasmic Reticulum (ER) \shortcite{Theodoridou2013}. It is also frequently assumed that $Ca^{2+}$ oscillations are driven by the emptying of the ER $Ca^{2+}$ store and not by the $IP_3R$ dynamics. Here, we develop a new `gating' model for $Ca^{2+}$ signalling in fertilisation that {more accurately captures the open probability of the $IP_3R$ dynamics, as a function of $Ca^{2+}$ and $IP_3$,} as determined by \shortciteA{Mak1998}.
    
    To develop a detailed understanding of gating models, we first study the models of \shortciteA{atri} and \shortciteA{lirinzel}. Subsequently, we study \shortciteA{Mak1998}, which includes the most up-to-date experimental data on the $IP_3R$ dynamics. We also review how these data have been incorporated into a model by \shortciteA{swedish}, though the latter is not a model for $Ca^{2+}$ signalling in fertilisation.
    
 Our model combines features of the \citeA{atri} model with the $IP_3R$ data by \citeA{Mak1998}. It contains one ODE for $[Ca^{2+}]$ in the cytosol and another ODE for the percentage of non inactivated $IP_3R$. We perform linear stability analysis and solve the model numerically, varying $[IP_3]$ as the bifurcation parameter. This model accurately reproduces key experimental features, including the low frequency and large amplitude of $Ca^{2+}$ oscillations in fertilisation. The model also captures that frequency and amplitude of $Ca^{2+}$ oscillations increase as $[IP_3]$ is increased \shortcite{Sanders2018}.
 With this model, we hope to guide future experiments that could eventually improve  clinical practice in In-Vitro Fertilisation. 
