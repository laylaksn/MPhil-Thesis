\chapter{Summary, Conclusions and Further work}
In this project we studied several $Ca^{2+}$ signalling models, focusing on the $IP_3R$ dynamics as determined in the experiments of \citeA{Mak1998}. In Chapter 1 we introduced the basic biological processes of $Ca^{2+}$ signalling, focusing on those of a mammalian egg at fertilisation. We also briefly reviewed the $Ca^{2+}$ models by \citeA{atri}, \citeA{lirinzel}, \citeA{dupontgoldbeter1993} and \citeA{deyoungkeizer}. Different models have been developed for different cell types. Many models and a substantial amount of experimental data are available for the Xenopus oocyte, which is similar to a mammalian egg. The most recent $Ca^{2+}$ models for fertilisation \cite{Theodoridou2013,Sanders2018,hofer} are not built with the correct $IP_3R$ dynamics determined by \citeA{Mak1998}. Many models rely on the refilling of the ER store to drive $Ca^{2+}$ oscillations, which is not the case for fertilisation \cite{Sanders2018, wakai}. We have thus developed a new gating model where we do not have $Ca^{2+}$ in the ER as a dynamic variable. This model reproduces key experimental features from \citeA{Sanders2018} and \citeA{Mak1998} as the frequency and amplitude of $Ca^{2+}$ oscillations increase as $[IP_3]$ increases. The model also uses the data and fitted equation from \citeA{Mak1998}, and relies on the $IP_3R$ dynamics to drive $Ca^{2+}$ oscillations.  

In Chapter 2, we analysed in detail two well-established gating models; the model by \citeA{atri} and the model by \citeA{lirinzel}. We paid close attention to key features, varying the $IP_3$ concentration as a bifurcation parameter and how the $Ca^{2+}$ behaviour changes accordingly. We studied in detail the bifurcation diagrams of both models \cite{kaouri,lirinzel}. We also looked in depth at the individual terms in these models, particularly those that describe the open probability for the $IP_3R$.

In Chapter 3 we studied the open probability equation by \citeA{Mak1998}. This involved studying how the three binding cites on an $IP_3R$ work (the activation of site 1 by $IP_3$, the activation of site 2 by $Ca^{2+}$, and the deactivation of site 3 by $Ca^{2+}$) work. These three binding processes occur with a certain probability, and these probabilities determine the overall probability of the $IP_3R$ being open and allowing $Ca^{2+}$ to pass through from the ER to the cytosol. The net flux of $Ca^{2+}$ out of the ER is determined by the number of open $IP_3R$. We examined the experimental data \cite{Mak1998} which were approximated by equation \eqref{foskett} derived for $P_O$. 

In Chapter 4 we studied the model by \citeA{swedish}. They used the equation for $P_O$ by \citeA{Mak1998}, but the latter model relied upon store depletion of the ER to drive oscillations, rather than the opening and closing of the $IP_3R$. We require the opening and closing of the $IP_3R$ to actually be the driving force for $Ca^{2+}$ oscillations in our model, as previously discussed and thus, the model by \citeA{swedish} is therefore not appropriate for $Ca^{2+}$ signalling in fertilisation.

In Chapter 5, we derived a new $Ca^{2+}$ signalling model for fertilisation, \eqref{newnew1}-\eqref{newnew2}. We used the Atri model \eqref{origatri1st}-\eqref{origatri2nd}, and the open probability equation for the $IP_3R$, $P_O$, by \citeA{Mak1998}, \eqref{foskett}. We implemented $P_O$ into the Atri model by splitting it into two terms $P_{O1}$ and $P_{O2}$, given in equation \eqref{foskett}, and substituted these into the Atri system in the place of the corresponding binding probabilities. $P_{O1}$ represents the probability of $Ca^{2+}$ binding to the activation site on the $IP_3R$, and $P_{O2}$ represents the probability of $Ca^{2+}$ binding to its inhibitory site. We also used the
half-activation constant in $J_{pump}$ as in the \citeA{hofer} model. Our model reproduced
the low frequency, large amplitude oscillations characteristic of fertilising mammalian eggs. We then performed simulations with increasing $IP_3$ concentration and parameter values as in Table \ref{tablewithhofersquared}. We showed that the oscillation amplitude and frequency increase as $IP_3$ concentration increases. The results are in agreement, qualitatively, with experiments carried out by \citeA{sneyd} and \citeA{Sanders2018}. The frequency of $Ca^{2+}$ oscillations increases as $[IP_3]$ increases, as shown in Figure \ref{tdd}. \citeA{sneyd} and \citeA{Sanders2018} show what happens when a large amount of $IP_3$ is suddenly released in the cell. These data were also in agreement with how our system behaves. Unfortunately, the range of $IP_3$ giving rise to $Ca^{2+}$ oscillations in our model is between $p=0.0085$ and $p=0.014$. This should be adjusted to be between approximately $p=0.01$ and $p>1$ ($\mu M$) \cite{Mak1998, karl} using a non-linear scaling. A scaling is also needed for $Ca^{2+}$ to oscillate between resting levels at $0.1 \mu M$, and peaks at $1 \mu M$, but this is outside the scope of this work.

In further work, one could incorporate a third ODE for $IP_3$ in a new model. We have introduced the correct $IP_3R$ dynamics into our model and tested it with constant $IP_3$. Evidence from \citeA{Sanders2018} suggests that there is a positive feedback of $Ca^{2+}$ on $PLC_{\zeta}$, and this leads to oscillations of $IP_3$. In Appendix A we briefly looked into modelling this and studied higher dimensional models by \citeA{sneyd}. {Once a third ODE for $IP_3$ has been incorporated, our new model could be used to make novel predictions for future experiments. }

Furthermore, stochastic modelling can be a direction for further work. As previously discussed, $Ca^{2+}$ signalling is intrinsically a stochastic process, so it is logical to develop stochastic models. The $IP_3R$ either open or close and these states change depending on random fluctuations induced by thermal noise. When many $IP_3R$ open, a global spike of $Ca^{2+}$ can occur (page 98 in \citeA{dupont}). Furthermore, one could look at modelling spatially extended systems (PDEs) rather than spatially homogeneous (ODEs). Cells are spatially distributed and $Ca^{2+}$ concentration varies across the cell.
